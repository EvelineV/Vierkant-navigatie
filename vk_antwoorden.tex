\begin{Antwoord}{1.1b}
            Je moet een uur terug tellen van de tijd die het nu is.
        
\end{Antwoord}
\begin{Antwoord}{1.3}
         Bissectrice
    
\end{Antwoord}
\begin{Antwoord}{1.5}
        Om 12 uur, ofwel het middaguur, staat de Zon recht in het Zuiden. \textcolor{red}{aanvullen}
    
\end{Antwoord}
\begin{Antwoord}{1.6}
        Dicht bij de evenaar zijn de schaduwen korter en lijkt de zon altijd recht boven je te staan (en door de tilt van de aardas - $23.5^{\circ}$ werkt deze methode niet tussen de Kreeftskeerkring en de Steenbokskeerkring).
    
\end{Antwoord}
\begin{Antwoord}{1.7}
        Bepaal de kleinste hoek tussen de kleine wijzer en de 11. De bissectrice van deze hoek wijst nu recht naar het zuiden.
    
\end{Antwoord}
\begin{Antwoord}{1.8}
        Je hoeft geen rekening meer te houden met zomertijd, de zon staat lager dus je kan makkelijker de richting van de schaduwen bepalen.
    
\end{Antwoord}
\begin{Antwoord}{1.10}
        Richt de 12 van je horloge naar de zon, de bissectrice tussen de 12 en de kleine wijzer wijst nu naar het Noorden.
    
\end{Antwoord}
\begin{Antwoord}{2.3a}
			12800 km * $\pi$ = 40200 km
		
\end{Antwoord}
\begin{Antwoord}{2.3b}
			1/360 * 40200 km = 112 km
		
\end{Antwoord}
\begin{Antwoord}{2.3c}
			24500 km
		
\end{Antwoord}
\begin{Antwoord}{2.3d}
			68 km
		
\end{Antwoord}
\begin{Antwoord}{2.4}
		Ja.
	
\end{Antwoord}
\begin{Antwoord}{2.5a}
			Nee, de afstand tussen meridianen neemt af naarmate je verder van de evenaar komt en wordt 0 bij de polen.
		
\end{Antwoord}
\begin{Antwoord}{2.5b}
			Afstanden die gelijk zijn in de vlakke projectie zijn niet noodzakelijkerwijs gelijk op het boloppervlak. De projectie is dus niet afstand-behoudend.
		
\end{Antwoord}
\begin{Antwoord}{2.6a}
			Z (zuid)
		
\end{Antwoord}
\begin{Antwoord}{2.6b}
			ZW (zuidwest)
		
\end{Antwoord}
\begin{Antwoord}{2.6c}
			$67.5\degree$
		
\end{Antwoord}
\begin{Antwoord}{2.13}
		Het antwoord is niet relevant. Verderop werken de deelnemers met een projectie die grootcirkels op rechte lijnen afbeeldt en kunnen ze de kortste afstand direct bepalen door een rechte lijn te trekken op die kaart. Dit kan dan vergeleken worden met hun eerste gok in deze opgave. Oplettende deelnemers zullen nu al opmerken dat het kortste pad geen rechte lijn is op de Mercator-projectie, omdat rechte lijnen loxodromen voorstellen en loxodromen in het algemeen niet samenvallen met grootcirkels.
	
\end{Antwoord}
\begin{Antwoord}{2.14}
		De evenaar.
	
\end{Antwoord}
\begin{Antwoord}{2.15}
		De kortste route tussen twee punten is eenvoudig te bepalen met behulp van een gnomonische projectie.
	
\end{Antwoord}
\begin{Antwoord}{4.1}
		Nee, als \'e\'en van de personen stil staat, kan de ander in een cirkel rondlopen.
	
\end{Antwoord}
\begin{Antwoord}{4.2}
		Alleen als de touwen exact in elkaars verlengde liggen. Als dat niet zo is, dan heeft het centrale punt de vrijheid om zich te bewegen in een cirkel rondom de rechte lijn tussen beide uiteinden. Als deze beweging zich tot twee dimensies beperkt, dan zijn er 2 punten waar de centrale persoon kan staan.
	
\end{Antwoord}
\begin{Antwoord}{4.3}
		Als de touwen allemaal in hetzelfde vlak liggen, dat wil zeggen, als alle uiteindes op min of meer dezelfde hoogte worden 	vastgehouden, dan wel. Maar oplettende deelnemers merken wellicht op dat als de centrale persoon lager of hoger is dan de 			andere drie (bijvoorbeeld omdat deze bukt of op een meubelstuk staat), er nog een andere positie is waarin alle touwen 				strak staan.
	
\end{Antwoord}
\begin{Antwoord}{4.11}
		Met \'e\'en satelliet te weinig zijn er 2 mogelijke locaties waar de ontvanger zich zou kunnen bevinden. Maar in vrijwel alle gevallen bevindt slechts 1 van de 2 mogelijke locaties zich op het aardoppervlak. Het andere punt kan uitgesloten worden omdat het zich diep onder de grond of hoog in de lucht bevindt.
	
\end{Antwoord}
\begin{Antwoord}{4.13}
		Er zijn meerdere mogelijkheden. Een mogelijkheid is dat de satelliet op vooraf vastgestelde tijdstippen een signaal verstuurt (bijvoorbeeld iedere seconde). Een andere mogelijkheid is dat het tijdstip van verzenden als boodschap met het signaal mee wordt gestuurd. In alle gevallen is het essentieel dat de GPS ontvanger een nauwkeurige klok heeft.
	
\end{Antwoord}
\begin{Antwoord}{4.15}
		In plaats van de co\"ordinaten in drie-dimensionale ruimte, moeten de co\"ordinaten in vier-dimensionale ruimte-tijd worden bepaald. Net zoals de overstap van 2 naar 3 dimensies betekent dat het minimale aantal bekende punten stijgt van 3 naar 4, zorgt de stap van 3 naar 4 dimensies ervoor dat het minimale aantal punten stijgt naar 5. Er zijn dus 5 satellieten nodig voor een nauwkeurige plaatsbepaling als de ontvanger geen nauwkeurige klok heeft (zonder gebruik te maken van extra aannames zoals dat de ontvanger zich dicht bij het aardoppervlak bevindt).
	
\end{Antwoord}
