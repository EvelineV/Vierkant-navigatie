\chapter{Astrolabium}

In dit hoofdstuk ga je aan de slag met het \textit{astrolabium}, een van de oudste navigatie-instrumenten. Je gebruikt het astrolabium zoals in figuur \ref{astrolabe-face}. Er bestaan twee varianten van het astrolabium, het sterrenkundig astrolabium en het zeemansastrolabium, dat versimpeld was en makkelijker te gebruiken op het dek van een schip. Het astrolabium dat wij gebruiken is een sterrenkundig astrolabium.

\begin{figure}
 \includegraphics[width=0.7\textwidth]{astrolabe-hi.png}
 \label{astrolabe-face}
\end{figure}

Je bouwt eerst je eigen astrolabium, daarvoor krijg je een werkblad met de onderdelen van het astrolabium uitgedeeld. Allereerst kijken we naar de achterkant van het astrolabium (figuur \ref{astrolabe-back}).

\begin{figure}
 \includegraphics[width=0.7\textwidth]{astrolabiumNL/mother_back.eps}
 \label{astrolabe-back}
 \caption{De achterkant van het astrolabium}
\end{figure}

De ringen, van buiten naar binnen, geven het volgende weer:
\begin{itemize}
 \item Schaal om de hoogte van een ster te bepalen.
 \item Dagen van de sterrenbeelden.
 \item Namen van de sterrenbeelden (de tekens van de dierenriem).
 \item Kalender passend bij het jaar 1394 (deze loopt negen dagen voor op de kalender voor onze tijd).
 \item Kalender passend bij onze eigen tijd. Deze kun je inkleuren met kleurpotlood of highlighter om aan te geven dat dit de kalender is die je moet gebruiken.
 \item In de volgende twee ringen worden traditioneel de naamdagen van katholieke of anglicaanse heiligen weergegeven. Deze zijn weggelaten in ons astrolabium.
 \item De binnenste ring (schaduwschaal) zullen we niet gebruiken.
\end{itemize}

De wijzer (alidade) (figuur \ref{alidade}) draait over de achterkant van het astrolabium. \textcolor{red}{Naast de alidade op je werkblad wordt ook de \textit{rule} afgebeeld. Deze zullen we niet gebruiken.}

\begin{figure}
 \includegraphics[width=0.7\textwidth]{astrolabiumNL/alidade}
 \label{alidade}
 \caption{De alidade}
\end{figure}

De voorkant van het astrolabium heeft het alfabet en een gradenboog rondom de \textit{plaat}. De plaat is gemaakt voor 52$^\circ$ Noorderbreedte, de breedtegraad van Nederland. Voor een andere breedtegraad heb je een andere plaat nodig. Het alfabet geeft de tijd aan, het kruis \kreuz is middernacht, de A is \'e\'en uur, de M is 12 uur 's middags, enzovoort.

Op de plaat zie je cirkels rondom een punt dat niet het middelpunt van het astrolabium is. Dit punt heet het \textit{zenit}, en het geeft het punt recht boven je hoofd (als je rechtop staat) weer. Eromheen zijn de cirkels genaamd \textit{almucantaren}. De almucantaar die gemarkeerd is met $0^\circ$ is de \textit{horizon}. De verticale lijn tussen de M en \kreuz is de noord-zuid meridiaan (het zuiden ligt bij M), en de lijn tussen S en F is de oost-west lijn.

Het laatste onderdeel, gemaakt van doorzichtig plastic, maar in vroeger tijden vaak een gouden of koperen plaat met veel uitsparingen, is de \textit{spin}. Op de spin staan een sterrenkaart en uren afgebeeld. De cirkel met daarin de namen van de sterrenbeelden is het \textit{zonnepad}. Ook staan er drie cirkels op afgebeeld. De buitenste (aan de rand van de spin) geeft de Steenbokskeerkring aan (dit is een parallel op $23.5^{\circ}$ zuiderbreedte).  De kleinere cirkel is de evenaar, en de kleinste cirkel (gedeeltelijk bedekt door het zonnepad) is de Kreeftskeerkring (de parallel op $23.5^{\circ}$ noorderbreedte). Als je de spin met de klok mee draait, zie je de sterren opkomen in het oosten en ondergaan in het westen.

De voorkant en de achterkant zijn niet helemaal cirkelvormig: aan de bovenkant zit nog een ring waaraan je het astrolabium kan vasthouden.

\begin{opgave}[\schaar]
 Knip nu alle onderdelen van het astrolabium uit. Lijm de voorkant en de achterkant op elkaar zodat de ring van beide onderdelen op elkaar zit, en je alle tekst kan lezen. Met een splitpen bevestig je de alidade op de achterkant en de spin op de voorkant van het astrolabium.
\end{opgave}

\subsection*{Zomertijd, wintertijd en lokale tijd}


\section{Tijdsbepaling met het astrolabium}

Op de spin zie je een ring met de namen van de twaalf sterrenbeelden van de dierenriem. Ieder sterrenbeeld hoort bij bepaalde data in het jaar:

\begin{center}
\begin{tabular}{|l|l|}
 \hline
 Sterrenbeeld & Data \\
 \hline
 Ram & 21 maart -- 21 april \\
 Stier & 21 april -- 21 mei \\
 Tweelingen & 21 mei -- 21 juni \\
 Kreeft & 21 juni -- 21 juli \\
 Leeuw & 21 juli -- 21 augustus \\
 Maagd & 21 augustus -- 21 september \\
 Weegschaal & 21 september -- 21 oktober
 Schorpioen & 21 oktober -- 21 november \\
 Boogschutter & 21 november -- 21 december \\
 Steenbok & 21 december -- 21 januari \\
 Waterman & 21 januari -- 21 februari \\
 Vissen & 21 februari -- 21 maart \\
 \hline
\end{tabular}
\end{center}

Je hoeft deze tabel niet uit je hoofd te leren, je kan namelijk op de achterkant van het astrolabium zien welk sterrenbeeld bij welke datum hoort. Let op dat je de binnenste kalender gebruikt! We gaan nu eerst het tijdstip van zonsopgang en zonsondergang van vandaag bepalen. 
\begin{opgave}
 Gebruik een stift of highlighter om, op het zonnepad (op de spin) een stip te zetten bij de datum van vandaag. Dit stipje is de zon. Nu draai je de spin zodat de zon op de oostelijke horizon ligt (dus op de linkerhelft van de plaat). De lijn die door het midden van de plaat en de zon gaat wijst nu een letter op de buitenrand van de plaat aan. Deze letter geeft het uur van zonsopgang aan. Hoe laat is dit?
 \begin{hint}
  Kamp is 13 tot 17 augustus 2018, je zet dus een stipje tussen het midden en het einde van de Leeuw op het zonnepad. Vergeet niet om te rekenen naar zomertijd.
 \end{hint}
 \begin{antwoord}
  $6:24$.
 \end{antwoord}
\end{opgave}

\begin{opgave}
 Bereken nu op dezelfde manier het tijdstip van zonsondergang vandaag. Gebruik hiervoor de westelijke horizon (de rechterkant van het astrolabium). 
 \begin{antwoord}
  $21:05$.
 \end{antwoord}
\end{opgave}

We gaan nu het tijdstip bepalen. Hiervoor is het belangrijk dat je niet recht in de zon kijkt!
\begin{opgave}
 Sta met je rug naar de zon toe. Houd het astrolabium aan de ring vast, voor je, zodat het vrij hangt. Draai de wijzer zodat het evenwijdig is met de zonnestralen - de flapjes mogen geen schaduw geven. Nu kun je op de buitenste schaalverdeling de hoogte van de zon aflezen. 
 \begin{subopgave}
 Wat is de hoogte van de zon?
 \end{subopgave}
 Draai nu de spin zo dat het stipje op de almucantaar met dezelfde hoogte staat. Bepaal nu de hoek tussen de zon en de noord-zuid meridiaan. Het tijdstip wordt nu gegeven door de volgende formule:
 \[ \textrm{Tijdstip} = 12 \pm \frac{24\times\textrm{hoek}}{360} \]
 waarbij de $\pm$ betekent dat je moet optellen als het middag is, maar aftrekken als het ochtend is.
 \begin{subopgave}
  Hoe laat is het volgens deze formule?
 \end{subopgave}
 \begin{subopgave}
  Reken dit om naar zomertijd. Klopt het met de tijd op de klok?
 \end{subopgave}
\end{opgave}

\begin{opgave}[\discussie]
 Je hebt in de vorige opgave de hoek tussen de zon en het zuiden bepaald. Wijs nu de richting van het zuiden aan. Ben je het eens met je groepje? Verschillen jullie antwoorden meer of minder dan in het horlogepracticum?
\end{opgave}
