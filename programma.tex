% Template voor een kampprogramma van de
% Stichting Vierkant voor Wiskunde. Copyright 2000-2005,2011-2012:
% M.P. Alberts, J.C.C. Langeveld, M. Hendriks, W.J. Palenstijn.
% Meest recente update: 1-6-2012

% De 'begeleider' optie produceert een begeleidersversie.
% De 'deelnemer' optie produceert een deelnemersversie (zonder antwoorden).
% Zonder 'begeleider' en 'deelnemer' wordt een voorlopige versie gemaakt,
%  met antwoorden, en met de huidige datum aangegeven.

%\documentclass[begeleider]{kampprogramma}
%\documentclass[deelnemer]{kampprogramma}
\documentclass{kampprogramma}

\usepackage[utf8]{inputenc}
%\usepackage[dutch]{babel}
\usepackage{color}
\usepackage{hyperref}
\usepackage{url}
\usepackage{graphics}


\titel{Navigatie}

\auteurs{
Eveline Vissee\\
Gideon Wormeester\\
}

% ------- Hier de naam van het (.eps) bestand met de kaftillustratie
% Je kan dit commando ook weglaten als je nog geen plaatje hebt.
\illustratie{vierkantlogo}
% -------

% ------- Hier het kamp waarvoor het programma is. (A/B/C + jaar)
\kamp{Zomerkamp B 2018}
% -------


\begin{document}

\voorpagina

\voorwoord
hallo!

% ------- Legenda
%
% In de legenda worden symbolen voor opgaven uitgelegd.
% Alleen de daadwerkelijk in het programma gebruikte symbolen worden uitgelegd.
%
% De symbolen hebben de volgende betekenis:
% \ster : moeilijke opgave
% \vinger : deze opgave heeft een hint
% \schaar : dit is een praktische opdracht
% \gr : je mag/moet een grafische rekenmachine gebruiken voor deze opgave
% \discussie : dit is een discussie-opgave
%
% Zie het stukje 'Opgave' hieronder over hoe je de symbolen bij een opgave zet

\legenda


\inhoudsopgave

% ------- Hier begint het eigenlijke programma

\chapter{Horloges}

We beginnen dit kampprogramma met een klein practicum. Gebruik hiervoor een horloge met wijzerplaat (misschien heb je er een bij je, misschien kun je je smartwatch instellen om een wijzerplaat te gebruiken). Anders kun je je begeleider vragen om het horlogewerkblad \textcolor{red}{HORLOGEWERKBLAD met lege wijzerplaat - iedere deelnemer heeft twee stuks nodig}.

Het doel van dit practicum is om het Noorden te bepalen. Als je je horloge gebruikt, zet het dan eerst gelijk met de wintertijd.

\begin{opgave}
    \begin{subopgave}
        Hoe laat is het nu?
    \end{subopgave}
    \begin{subopgave}
        Hoe laat is het nu -- in wintertijd?
        \begin{hint}
            In maart gingen alle klokken een uur vooruit. In oktober gaan ze een uur terug. 
        \end{hint}
        \begin{antwoord}
            Je moet een uur terug tellen van de tijd die het nu is.
        \end{antwoord}
    \end{subopgave}
\end{opgave}

Als je het werkblad gebruikt, teken dan eerst de juiste stand van de wijzers op de wijzerplaat - in de wintertijd.

\textbf{LET OP:} kijk nooit rechtstreeks naar de zon! Ook niet als je een zonnebril draagt. Het is beter om te kijken naar de richting van de schaduwen.

\textcolor{red}{gebruik stok/parasol als zonnewijzer?}

Zorg dat de kleine wijzer van je horloge in de richting van de zon wijst. Op je werkblad, markeer de kleinste hoek tussen de kleine wijzer en de 12. Teken vervolgens de lijn die deze hoek precies in twee\"{e}n deelt.

\begin{opgave}
    Hoe wordt deze lijn ook wel genoemd? 
    \begin{antwoord}
         Bissectrice
    \end{antwoord}
\end{opgave}

Deze lijn wijst precies naar het Zuiden. Teken nu op je werkblad de richting van het Noorden.

\begin{opgave}
    Hoe weet je dat deze lijn precies naar het Zuiden wijst?
    \begin{antwoord}
        om 12 uur, ofwel het middaguur, staat de Zon recht in het Zuiden. \textcolor{red}{aanvullen})
    \end{antwoord}
\end{opgave}

\begin{opgave}
    Waarom werkt deze methode minder goed als je dichter bij de evenaar bent, en beter als je dichter bij de Noordpool bent? 
    \begin{antwoord}
        Dicht bij de evenaar zijn de schaduwen korter en lijkt de zon altijd recht boven je te staan (en door de tilt van de aardas - $23.5^{\circ}$ werkt deze methode niet tussen de Kreeftskeerkring en de Steenbokskeerkring).
    \end{antwoord}
\end{opgave}

\begin{opgave}
    Hoe pas je deze methode aan voor de zomer? Zet je horloge terug op zomertijd en probeer of je dezelfde uitkomst krijgt.
    \begin{antwoord}
        bepaal de kleinste hoek tussen de kleine wijzer en de 11. De bissectrice van deze hoek wijst nu recht naar het zuiden.   
    \end{antwoord}
\end{opgave}

\begin{opgave}
    Waarom werkt deze methode beter in de winter dan in de zomer? 
    \begin{antwoord}
        je hoeft geen rekening meer te houden met zomertijd, de zon staat lager dus je kan makkelijker de richting van de schaduwen bepalen.
    \end{antwoord}
\end{opgave}

\begin{opgave}
    De offici\"ele tijd hier in Nederland is hetzelfde als in het westelijkste puntje van Spanje (dat twee tijdzones verder naar het westen ligt), maar ook hetzelfde als in het oostelijkste puntje van Polen (dat twee tijdzones verder naar het oosten ligt). \textcolor{red}{INVOEGEN: KAARTJE VAN EUROPA MET TIJDZONES} Wat voor problemen levert dit op? Hoe pas je de methode aan?
\end{opgave}

\begin{opgave}
    Hoe zou je deze methode aanpassen als je in Australi\"{e} was?
    \begin{antwoord}
        Richt de 12 van je horloge naar de zon, de bissectrice tussen de 12 en de kleine wijzer wijst nu naar het Noorden.
    \end{antwoord}
\end{opgave}

\chapter{Je plaats op een ronde Aarde}

\section{Inleiding}

Wanneer je opzoekt waar je bent of waar je naar toe gaat, kijk je vrijwel altijd op een kaart. Of je dit nu doet met een papieren kaart, een informatiebord langs de weg of met een app op je telefoon, \'e\'en ding hebben ze gemeen: De kaart is plat. Maar de Aarde is alles behalve plat en het vertalen van de ronde aarde naar een platte kaart is niet eenvoudig.

In dit hoofdstuk ga je leren wat er bij komt kijken om te navigeren op een ronde wereld en hoe het toch, met enige beperkingen, mogelijk is om platte kaarten te gebruiken voor een ronde Aarde.

\section{Co\"ordinaten op een boloppervlak}

%\includegraphics[draft]{Carthesisch}

Hierboven zie je een voorbeeld van een plat rooster. Misschien heb je dit al eerder gezien tijdens de wiskundeles. Zo'n rooster zou je kunnen gebruiken om over een kaart heen te leggen. Langs de \textit{x}-as en \textit{y}-as staan getallen waarmee we de co\"ordinaten van punten in het rooster kunnen bepalen. Je kunt het punt met \textit{x} = 2 en \textit{y} = 3 vinden door op de \textit{x}-as de waarde 2 op te zoeken en op de \textit{y}-as de waarde 3. Het gezochte punt is vervolgens te vinden op de plek waar de 2 lijnen vanuit de gevonden punten op de assen elkaar snijden. Een korte notatie voor de co\"ordinaten van dit punt is $(2, 3)$.

\begin{opgave}
	\begin{subopgave}
		Waar ligt het punt (4, 1) in bovenstaand rooster?
	\end{subopgave}
	\begin{subopgave}
		Wat zijn de co\"ordinaten van het in het bovenstaand rooster aagegeven punt P?
	\end{subopgave}
\end{opgave}





\chapter{Astrolabium}

In dit hoofdstuk ga je aan de slag met het \textit{astrolabium}, een van de oudste navigatie-instrumenten. Je gebruikt het astrolabium zoals in figuur \ref{astrolabe-face}.

\begin{figure}
 \includegraphics[width=0.7\textwidth]{astrolabe-hi.png}
 \label{astrolabe-face}
\end{figure}

Je bouwt eerst je eigen astrolabium, daarvoor krijg je een werkblad met de onderdelen van het astrolabium uitgedeeld. Allereerst kijken we naar de achterkant van het astrolabium (figuur \ref{astrolabe-back}).

\begin{figure}
 \includegraphics[width=0.7\textwidth]{astrolabiumNL/mother_back}
 \label{astrolabe-back}
 \caption{De achterkant van het astrolabium}
\end{figure}

De ringen, van buiten naar binnen, geven het volgende weer:
\begin{itemize}
 \item Schaal om de hoogte van een ster te bepalen.
 \item Dagen van de sterrenbeelden.
 \item Namen van de sterrenbeelden (de tekens van de dierenriem).
 \item Kalender passend bij het jaar 1394 (deze loopt negen dagen voor op de kalender voor onze tijd).
 \item Kalender passend bij onze eigen tijd.
 \item In de volgende twee ringen worden traditioneel de naamdagen van katholieke of anglicaanse heiligen weergegeven. Deze zijn weggelaten in ons astrolabium.
 \item De binnenste ring (schaduwschaal) zullen we niet gebruiken.
\end{itemize}

De alidade (figuur \ref{alidade}) draait over de achterkant van het astrolabium. \textcolor{red}{Naast de alidade op je werkblad wordt ook de \textit{rule} afgebeeld. Deze zullen we niet gebruiken.}

\begin{figure}
 \includegraphics[width=0.7\textwidth]{astrolabiumNL/alidade}
 \label{alidade}
 \caption{De alidade}
\end{figure}

De voorkant van het astrolabium heeft het alfabet en een gradenboog rondom de \textit{plaat}. De plaat is gemaakt voor 52$^\circ$ Noorderbreedte, de breedtegraad van Nederland. Voor een andere breedtegraad heb je een andere plaat nodig. Het alfabet geeft de tijd aan, het kruis \kreuz is middernacht, de A is \'e\'en uur, de M is 12 uur 's middags, enzovoort.

Het laatste onderdeel, gemaakt van doorzichtig plastic, maar in vroeger tijden vaak een gouden of koperen plaat met veel uitsparingen, is de \textit{spin}. Op de spin staan een sterrenkaart en uren afgebeeld. 

De voorkant en de achterkant zijn niet helemaal cirkelvormig: aan de bovenkant zit nog een ring waaraan je het astrolabium kan vasthouden.

\begin{opgave}[\schaar]
 Knip nu alle onderdelen van het astrolabium uit. Lijm de voorkant en de achterkant op elkaar zodat de ring van beide onderdelen op elkaar zit, en je alle tekst kan lezen. Met een splitpen bevestig je de alidade op de achterkant en de spin op de voorkant van het astrolabium.
\end{opgave}


%\section{Avondpracticum}

%\section{GPS}

% ------- Hier eindigt het eigenlijke programma

\hintsantwoorden

\end{document}
