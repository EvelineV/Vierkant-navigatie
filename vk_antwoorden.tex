\begin{Antwoord}{2.3a}
			12800 km * $\pi$ = 40200 km
		
\end{Antwoord}
\begin{Antwoord}{2.3b}
			1/360 * 40200 km = 112 km
		
\end{Antwoord}
\begin{Antwoord}{2.3c}
			24500 km
		
\end{Antwoord}
\begin{Antwoord}{2.3d}
			68 km
		
\end{Antwoord}
\begin{Antwoord}{2.4}
		Ja.
	
\end{Antwoord}
\begin{Antwoord}{2.5a}
			Nee, de afstand tussen meridianen neemt af naarmate je verder van de evenaar komt en wordt 0 bij de polen.
		
\end{Antwoord}
\begin{Antwoord}{2.5b}
			Afstanden die gelijk zijn in de vlakke projectie zijn niet noodzakelijkerwijs gelijk op het boloppervlak. De projectie is dus niet afstand-behoudend.
		
\end{Antwoord}
\begin{Antwoord}{4.1}
		Nee, als \'e\'en van de personen stil staat, kan de ander in een cirkel rondlopen.
	
\end{Antwoord}
\begin{Antwoord}{4.2}
		Alleen als de touwen exact in elkaars verlengde liggen. Als dat niet zo is, dan heeft het centrale punt de vrijheid om zich te bewegen in een cirkel rondom de rechte lijn tussen beide uiteinden. Als deze beweging zich tot twee dimensies beperkt, dan zijn er 2 punten waar de centrale persoon kan staan.
	
\end{Antwoord}
\begin{Antwoord}{4.3}
		Als de touwen allemaal in hetzelfde vlak liggen, dat wil zeggen, als alle uiteindes op min of meer dezelfde hoogte worden 	vastgehouden, dan wel. Maar oplettende deelnemers merken wellicht op dat als de centrale persoon lager of hoger is dan de 			andere drie (bijvoorbeeld omdat deze bukt of op een meubelstuk staat), er nog een andere positie is waarin alle touwen 				strak staan.
	
\end{Antwoord}
\begin{Antwoord}{4.11}
		Met \'e\'en satelliet te weinig zijn er 2 mogelijke locaties waar de ontvanger zich zou kunnen bevinden. Maar in vrijwel alle gevallen bevindt slechts 1 van de 2 mogelijke locaties zich op het aardoppervlak. Het andere punt kan uitgesloten worden omdat het zich diep onder de grond of hoog in de lucht bevindt.
	
\end{Antwoord}
\begin{Antwoord}{4.13}
		Er zijn meerdere mogelijkheden. Een mogelijkheid is dat de satelliet op vooraf vastgestelde tijdstippen een signaal verstuurt (bijvoorbeeld iedere seconde). Een andere mogelijkheid is dat het tijdstip van verzenden als boodschap met het signaal mee wordt gestuurd. In alle gevallen is het essentieel dat de GPS ontvanger een nauwkeurige klok heeft.
	
\end{Antwoord}
\begin{Antwoord}{4.15}
		In plaats van de co\"ordinaten in drie-dimensionale ruimte, moeten de co\"ordinaten in vier-dimensionale ruimte-tijd worden bepaald. Net zoals de overstap van 2 naar 3 dimensies betekent dat het minimale aantal bekende punten stijgt van 3 naar 4, zorgt de stap van 3 naar 4 dimensies ervoor dat het minimale aantal punten stijgt naar 5. Er zijn dus 5 satellieten nodig voor een nauwkeurige plaatsbepaling als de ontvanger geen nauwkeurige klok heeft (zonder gebruik te maken van extra aannames zoals dat de ontvanger zich dicht bij het aardoppervlak bevindt).
	
\end{Antwoord}
