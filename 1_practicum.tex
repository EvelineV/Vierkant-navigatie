\chapter{Horloges}

We beginnen dit kampprogramma met een klein practicum. Gebruik hiervoor een horloge met wijzerplaat (misschien heb je er een bij je, misschien kun je je smartwatch instellen om een wijzerplaat te gebruiken). Anders kun je je begeleider vragen om het horlogewerkblad \textcolor{red}{MAAK HORLOGEWERKBLAD met lege wijzerplaat - iedere deelnemer heeft twee stuks nodig}.

Het doel van dit practicum is om het Noorden te bepalen. Als je je horloge gebruikt, zet het dan eerst gelijk met de wintertijd.

\begin{itemize}
\item Hoe laat is het nu?
\item Hoe laat is het nu -- in wintertijd? (hint: in maart gingen alle klokken een uur vooruit. In oktober gaan ze een uur terug. antwoord: dat betekent dat je nu een uur terug moet tellen.)
\end{itemize}

Als je het werkblad gebruikt, teken dan eerst de juiste stand van de wijzers op de wijzerplaat - in de wintertijd.

\textbf{LET OP:} kijk nooit rechtstreeks naar de zon! Ook niet als je een zonnebril draagt. Het is beter om te kijken naar de richting van de schaduwen.

\textcolor{red}{gebruik stok/parasol als zonnewijzer?}

Zorg dat de kleine wijzer van je horloge in de richting van de zon wijst. Op je werkblad, markeer de kleinste hoek tussen de kleine wijzer en de 12. Teken vervolgens de lijn die deze hoek precies in twee\"{e}n deelt.

\begin{itemize}
 \item Hoe wordt deze lijn ook wel genoemd? (antwoord: bissectrice)
 \item Deze lijn wijst precies naar het Zuiden. Teken nu op je werkblad de richting van het Noorden.
 \item Hoe weet je dat deze lijn precies naar het Zuiden wijst? (antwoord: om 12 uur, ofwel het middaguur, staat de Zon recht in het Zuiden. \textcolor{red}{aanvullen})
 \item Waarom werkt deze methode minder goed als je dichter bij de evenaar bent, en beter als je dichter bij de Noordpool bent? (antwoord: dicht bij de evenaar zijn de schaduwen korter en lijkt de zon altijd recht boven je te staan (en door de tilt van de aardas - $23.5^{\circ}$ werkt deze methode niet tussen de Kreeftskeerkring en de Steenbokskeerkring). 
 \item Hoe pas je deze methode aan voor de zomer? Zet je horloge terug op zomertijd en probeer of je dezelfde uitkomst krijgt. (antwoord: bepaal de kleinste hoek tussen de kleine wijzer en de 11.)
 \item Waarom werkt deze methode beter in de winter dan in de zomer? (antwoord: je hoeft geen rekening meer te houden met zomertijd, de zon staat lager dus je kan makkelijker de richting van de schaduwen bepalen, )
 \item De plaatselijke tijd hier in Nederland is hetzelfde als in het westelijkste puntje van Spanje (dat twee tijdzones verder naar het westen ligt), maar ook hetzelfde als in het oostelijkste puntje van Polen (dat twee tijdzones verder naar het oosten ligt). \textcolor{red}{KAARTJE VAN EUROPA MET TIJDZONES} Wat voor problemen levert dit op? Hoe pas je de methode aan?
 \item Hoe zou je deze methode aanpassen als je in Australi\"{e} was? (antwoord: richt de 12 van je horloge naar de zon, de bissectrice tussen de 12 en de kleine wijzer wijst nu naar het Noorden.)
\end{itemize}



