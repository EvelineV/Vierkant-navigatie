\chapter{Je plaats op een ronde Aarde}

\section{Inleiding}

Wanneer je opzoekt waar je bent of waar je naar toe gaat, kijk je vrijwel altijd op een kaart. Of je dit nu doet met een papieren kaart, een informatiebord langs de weg of met een app op je telefoon, \'e\'en ding hebben ze gemeen: De kaart is plat. Maar de Aarde is alles behalve plat en het vertalen van de ronde aarde naar een platte kaart is niet eenvoudig.

In dit hoofdstuk ga je leren wat er bij komt kijken om te navigeren op een ronde wereld en hoe het toch, met enige beperkingen, mogelijk is om platte kaarten te gebruiken voor een ronde Aarde.

\section{Co\"ordinaten op een boloppervlak}

%\includegraphics[draft]{Carthesisch}

Hierboven zie je een voorbeeld van een plat rooster. Misschien heb je dit al eerder gezien tijdens de wiskundeles. Zo'n rooster zou je kunnen gebruiken om over een kaart heen te leggen. Langs de \textit{x}-as en \textit{y}-as staan getallen waarmee we de co\"ordinaten van punten in het rooster kunnen bepalen. Je kunt het punt met \textit{x} = 2 en \textit{y} = 3 vinden door op de \textit{x}-as de waarde 2 op te zoeken en op de \textit{y}-as de waarde 3. Het gezochte punt is vervolgens te vinden op de plek waar de 2 lijnen vanuit de gevonden punten op de assen elkaar snijden. Een korte notatie voor de co\"ordinaten van dit punt is $(2, 3)$.

\begin{opgave}
	\begin{subopgave}
		Waar ligt het punt (4, 1) in bovenstaand rooster?
	\end{subopgave}
	\begin{subopgave}
		Wat zijn de co\"ordinaten van het in het bovenstaand rooster aagegeven punt P?
	\end{subopgave}
\end{opgave}

Rest van de sectie:

\begin{itemize}
	\item Introductie bolco\"ordinaten
	\item Incompatibiliteit tussen bolco\"ordinaten en Carthesische co\"ordinaten
\end{itemize}

\section{Kaartprojecties}

\begin{itemize}
	\item Equirectangulair (meridianen \& parallen worden rechte lijnen)
	\item Mercator (pad met constante koers is een rechte lijn)
	\item Gnomonic (grootcirkels zijn rechte lijnen)
\end{itemize}

\section{Kompas als navigatie instrument}

\begin{itemize}
	\item Intro kompas
	\item Verschil tussen constante kompas-koers en grootcirkel-pad
	\item Verschil tussen magnetische noorden \& ware noorden en kompas-correcties.
\end{itemize}

