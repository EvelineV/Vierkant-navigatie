\chapter{Moderne nagivatie}

Dit hoofdstuk zal vooral ingaan op de principes van trilateratie. De deelnemers vinden uit hoe trilateratie gebruikt kan worden om plaats te bepalen en hoeveel vaste punten (satellieten) nodig zijn voor plaatsbepaling. Daarnaast wordt ingegaan op het feit dat voor plaatsbepaling met GPS alle partijen een zeer accurate klok moeten hebben, wat doorgaans niet het geval is in een GPS-ontvanger.

\section{Touwtjes-trilateratie}

Praktische opdrachten waar een deelnemer \'e\'en of meerdere touwtjes van vaste lengte vasthoudt, terwijl de andere kant van de touwtjes worden vastgehouden door andere deelnemers/begeleiders. Doel is om het aantal vrijheidsgraden te bepalen als functie van het aantal satellieten.

\section{Snijdende Cirkels}

Een alternatieve methode om het trilateratie-probleem te illustreren. Hier worden de mogelijke posities bepaald aan de hand van snijpunten van cirkels (in 2D, of boloppervlakken in 3D). Deze sectie is individueel en deelnemers kunnen hier direct mee beginnen als onvoldoende anderen beschikbaar zijn voor de touwtjes.

\section{Plaatsbepaling zonder tijd?}

Bepalen van de afstand tussen ontvanger en GPS-satelliet vereist een zeer nauwkeurige klok (opgave: schat te verwachten fout ten gevolge van kleine onnauwkeurigheden van de klok van de ontvanger). Omdat ontvangers deze normaalgesproken niet hebben, moet de tijd bepaald worden uit signalen van de satellieten. In de praktijk betekent dit het toevoegen van een extra onbekende aan de plaatsbepalingsvergelijkingen en daarmee de noodzaak van een extra satelliet (= extra vergelijking) voor nauwkeurige plaatsbepaling.