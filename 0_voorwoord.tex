In dit onderzoeksprogramma zoek je uit hoe mensen vroeger, voordat ze over smartphones met GPS beschikten, de weg vonden. We gaan het hebben over het bepalen van het Noorden (of Zuiden), breedte- en lengtegraden. We werken met het astrolabium om de tijd te bepalen aan de hand van de sterren.

Al in de Oudheid reisden handelaars de hele (bekende) wereld over met hun goederen; wat van ver komt is bijzonder en daardoor waardevol. Om op de juiste bestemming te komen is het niet praktisch om de weg uit je hoofd te leren, zeker niet als de reis maanden of jaren kan duren. De oudst gevonden kaarten komen dan ook uit de 25$^{e}$ eeuw voor het begin van onze jaartelling. 

Vroege schipskapiteins volgden vaak de kust om hun weg te vinden, maar dit was niet altijd genoeg. Om van Kreta naar Egypte te komen kostte meer dan een dag, waarbij het schip 's nachts niet in zicht van land was, en de bemanning dus de sterren moest gebruiken om het schip in de juiste richting te sturen. Zeekaarten werden gemaakt waarop ondieptes en opvallende kenmerken van de kust werden aangegeven.

Een andere vroege uitvinding was het peillood. Een zwaar gewicht dat achter het schip in het water gelaten werd kon gebruikt worden om de diepte van het water te bepalen, en daarmee kon de afstand tot het dichtstbijzijnde land worden geschat. Ook de wind kon helpen bij het bepalen van de richting.

De hemel was echter de meest nauwkeurige navigatiemethode op open zee. De Grieks-Egyptische geleerde Eratosthenes berekende in de tweede eeuw voor het begin van onze jaartelling de omtrek van de aarde door gebruik te maken van de stand van de zon. Zijn berekening was nauwkeuriger (en maar 10$\%$ groter dan de werkelijke waarde) dan die van Christoffel Columbus, meer dan 1500 jaar later.

's Nachts werd gebruik gemaakt van de sterren. De Poolster en de sterren van de Kleine Beer zijn heel geschikt om de richting van het Noorden te vinden, en het astrolabium, de sextant, octant en de jacobsstaf werden allen ontwikkeld om gebruik te maken van de sterren.


